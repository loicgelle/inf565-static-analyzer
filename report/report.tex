\title{INF565 Project report}
\documentclass[paper=a4, fontsize=11pt]{scrartcl}
\usepackage[utf8]{inputenc}
\usepackage[T1]{fontenc}
\usepackage{fourier}

\usepackage[francais]{babel}
\usepackage[protrusion=true,expansion=true]{microtype}	
\usepackage{amsmath,amsfonts,amsthm}
\usepackage{hyperref}
\usepackage{minted}
\usepackage{graphicx}

\usepackage{sectsty}
\allsectionsfont{\centering \normalfont\scshape}

\usepackage[nottoc, notlof, notlot]{tocbibind}

\usepackage{fancyhdr}
\pagestyle{fancyplain}
\fancyhead{}											% No page header
\fancyfoot[L]{}											% Empty 
\fancyfoot[C]{}											% Empty
\fancyfoot[R]{\thepage}									% Pagenumbering
\renewcommand{\headrulewidth}{0pt}			% Remove header underlines
\renewcommand{\footrulewidth}{0pt}				% Remove footer underlines
\setlength{\headheight}{13.6pt}

\numberwithin{figure}{section}			% Figurenumbering: section.fig#
\numberwithin{table}{section}				% Tablenumbering: section.tab#

\newcommand{\horrule}[1]{\rule{\linewidth}{#1}} 	% Horizontal rule

\title{	
		\usefont{OT1}{bch}{b}{n}
		\normalfont \normalsize \textsc{INF565 : Vérification} \\ [25pt]
		\horrule{0.5pt} \\[0.4cm]
		\huge Analyseur statique pour un sous-ensemble de Java \\
		\horrule{2pt} \\[0.5cm]
}
\author{
		\normalfont 								\normalsize
        Lo\"{i}c Gelle\\[-3pt]		\normalsize
        16 mars 2017
}
\date{}


%%% Begin document
\begin{document}
\maketitle
\section{Introduction}

Ce rapport a pour objet de présenter le travail effectué lors du projet du cours INF565 -- Vérification. La première partie sera consacrée à la description de la mise en place du projet et des premières vérifications effectuées. La partie suivante détaille les extensions qui ont été implémentées et leurs motivations.\\

Le code est disponible sur Github\footnote{Voir \url{https://github.com/loicgelle/inf565-static-analyzer}}. Des exemples sont fournis dans le dossier \texttt{code/examples/} du projet ; lorsque ce rapport fera référence à un fichier d'exemple, il le fera relativement à ce répertoire. Pour plus d'informations sur l'utilisation de l'analyseur et la structure du projet, on pourra se référer au fichier \texttt{README.md}.\\

\textit{\textbf{Important :} toutes les analyses statiques effectuées dans le projet -- sauf l'analyse de typage -- supposent que le programme consiste en une seule classe contenant une seule fonction. L'analyse interprocédurale n'a pas été envisagée.}

\section{Mise en place du projet et vérifications de base}

\subsection{Mise en place et premières analyses statiques}

Les premières étapes ont été l'implémentation d'une fonction d'affichage des programmes, à des fins de débogage, ainsi que d'un interpréteur. L'interprète suit simplement la structure de l'arbre syntaxique et modifie l'environnement -- une table de hachage qui associe à chaque variable sa valeur -- en conséquence.\\

L'interpréteur est capable de lever des exceptions en cas de problème de typage ou d'utilisation d'une variable non initialisée. Le programme \texttt{typing\_bad/binary\_op.java} suivant

\begin{minted}{java}
class A {
  static int x;
  static boolean b1, b2;

  static void main() {
    x = 1;
    b1 = true;
    b2 = x + b1;
  }
}
\end{minted}

renvoie une erreur à l'interprétation :

\begin{verbatim}
$ ./analyzer --interpret A examples/typing_bad/binary_op.java
examples/typing_bad/binary_op.java:8.9-15:
Interpretation error!
Cannot apply operator to non integer values
\end{verbatim}

De même, l'interpréteur échoue en cas d'utilisation d'une variable non initialisée, comme c'est le cas dans le fichier \texttt{init\_bad/condition.java} :

\begin{verbatim}
$ ./analyzer --interpret A examples/init_bad/condition.java
examples/init_bad/condition.java:6.4-8.5:
Interpretation error!
IF statement needs a boolean value to test
\end{verbatim}

Cependant, ces erreurs devraient plutôt être anticipées dès l'analyse statique. L'analyseur statique permet de détecter l'utilisation de variables non initialisées avant l'exécution :

\begin{verbatim}
$ ./analyzer examples/init_bad/condition.java
examples/init_bad/condition.java:6.8-10:
Error: use of not initialized variable
\end{verbatim}

L'analyseur détecte l'initialisation des variables lorsque les flots de contrôle sont un peu plus complexes, comme c'est le cas pour \texttt{init\_good/double\_condition.java}.\\

Une vérification statique des types a également été implémentée. Sur le programme mal typé présenté plus haut, l'analyseur détecte l'erreur de typage :

\begin{verbatim}
$ ./analyzer examples/typing_bad/binary_op.java
examples/typing_bad/binary_op.java:8.9-15:
Typing error
\end{verbatim}

\subsection{Analyses statiques plus poussées et simplifications}

La première analyse statique concernant la valeur des variables est une analyse de constantes. Elle est dans l'approche assez similaire à l'analyse des variables non initialisées. Si le mode de débogage est activé, avec l'option \texttt{-{}-debug}, l'analyseur imprime l'environnement du programme avant chaque instruction analysée. Dans tous les cas, il imprime à la fin l'environnement final ainsi que le programme obtenu après éventuelles simplifications.\\

Cette première analyse a été l'occasion de se pencher sur la gestion des boucles. Dans le cas de boucles courtes, l'analyseur termine et renvoie un environnement précis, comme c'est le cas pour le fichier \texttt{examples/analysis/short\_loop.java} :

\begin{verbatim}
$ ./analyzer --domain 0 examples/analysis/short_loop.java
FINAL STATE IS:
0 -> int: 7
1 -> int: 64
2 -> bool: false
3 -> bool: true
----------
Class name: A
Class body:
   Declaration of variable:
     Variable:
       name: x | type: int | uniqueId: 0
   Declaration of variable:
     Variable:
       name: y | type: int | uniqueId: 1
   Declaration of variable:
     Variable:
       name: b1 | type: bool | uniqueId: 2
   Declaration of variable:
     Variable:
       name: b2 | type: bool | uniqueId: 3
   Declaration of function:
     Proc name: main
     Proc body:
        x <- 1
        y <- 1
        while ((x < 7)) do:
            x <- (x + 1)
            y <- (y * 2)
        b1 <- false
        b2 <- true
------------
\end{verbatim}

Dans le cas de boucles plus "longues", l'analyse ne peut pas prendre le risque d'exécuter des centaines de tours de boucle, voire de boucler indéfiniment. A partir d'un certain nombre de tours de boucle exécutés par l'analyse abstraite, l'analyseur renvoie une surapproximation de l'état obtenu en sortie de boucle. Dans le cas d'une analyse de constante, il est forcé de considérer comme indéterminées toutes les variables qui sont modifiées dans le corps d'une boucle. Le programme précédent en version "longue", \texttt{examples/analysis/long\_loop.java}, est analysé de cette manière :

\begin{verbatim}
$ ./analyzer --domain 0 examples/analysis/long_loop.java
FINAL STATE IS:
0 -> int: undetermined
1 -> int: undetermined
2 -> bool: undetermined
3 -> bool: undetermined
----------
Class name: A
Class body:
   Declaration of variable:
     Variable:
       name: x | type: int | uniqueId: 0
   Declaration of variable:
     Variable:
       name: y | type: int | uniqueId: 1
   Declaration of variable:
     Variable:
       name: b1 | type: bool | uniqueId: 2
   Declaration of variable:
     Variable:
       name: b2 | type: bool | uniqueId: 3
   Declaration of function:
     Proc name: main
     Proc body:
        x <- 1
        y <- 1
        while ((x < 100)) do:
            x <- (x + 1)
            y <- (y * 2)
        if ((y < 1)) then:
            b1 <- true
            b2 <- false
        else:
            b1 <- false
            b2 <- true
------------
\end{verbatim}

On voit bien les limites de l'analyse de constantes sur un tel exemple. On aimerait pouvoir conclure que $y \geq 1$ en tout instant de l'exécution du programme et donc se débarrasser d'une partie de la structure conditionnelle finale.\\

L'analyse d'intervalles s'en sort mieux sur cet exemple. On obtient cette fois-ci :
\begin{verbatim}
$ ./analyzer --domain 1 examples/analysis/long_loop.java
FINAL STATE IS:
0 -> int: [1, +infty]
1 -> int: [1, +infty]
2 -> bool: false
3 -> bool: true
----------
Class name: A
Class body:
   Declaration of variable:
     Variable:
       name: x | type: int | uniqueId: 0
   Declaration of variable:
     Variable:
       name: y | type: int | uniqueId: 1
   Declaration of variable:
     Variable:
       name: b1 | type: bool | uniqueId: 2
   Declaration of variable:
     Variable:
       name: b2 | type: bool | uniqueId: 3
   Declaration of function:
     Proc name: main
     Proc body:
        x <- 1
        y <- 1
        while ((x < 100)) do:
            x <- (x + 1)
            y <- (y * 2)
        b1 <- false
        b2 <- true
------------
\end{verbatim}

L'analyseur comprend que les deux variables entières ne font que croître dans le corps de la boucle : il donne donc une surapproximation qui conserve toutefois une borne inférieure assez précise sur les variables. Il peut donc évaluer à \texttt{false} la condition en fin de programme, et stocke dans une table de hachage dédiée aux simplifications que la branche \texttt{true} de la structure conditionnelle est inutile ; une deuxième passe sur l'arbre syntaxique se sert des informations contenues dans cette table de hachage pour renvoyer un arbre syntaxique simplifié.\\

Ces premières analyses statiques de variables ont donné lieu à une première passe de refactorisation du code :

\begin{itemize}
\item La gestion de l'arbre syntaxique du programme est réalisée dans le fichier \newline \texttt{static\_analysis\_variables.ml} ;
\item Ce dernier utilise le foncteur contenu dans \texttt{abstract\_domain.ml}, lequel gère les environnements de variables -- intersections, fusions -- et les structures complexes du langage -- conditions, boucles ;
\item Enfin, les opérations primitives propres au domaine abstrait sont réalisées dans les modules \texttt{domain\_constants.ml} et \texttt{domain\_intervals.ml}, lesquels implémentent l'interface de module définie dans \texttt{domains.ml}.
\end{itemize}

\section{Extensions et raffinement de l'analyseur}

Les développement décrits précédemment ainsi que les tests réalisés ont mis en lumière des améliorations possibles de l'analyseur. Celles qui ont été implémentées sont évoquées dans les parties qui suivent.

\subsection{Refactorisation de l'analyse de booléens}

La gestion des types booléens a donné lieu à de la duplication de code dans les domaines abstraits des constantes et des intervalles. Chacun avait en effet un type d'information spécial pour les variables booléennes, qu'il gérait exactement de la même manière que l'autre.\\

Pour remédier à cela, un foncteur \texttt{domain\_wrapper\_boolean} a été développé. Il gère les variables booléennes lors de l'analyse, et passe la main au domaine abstrait actif lorsqu'il s'agit de gérer des informations sur les entiers.

\subsection{Meilleure gestion des structures conditionnelles}

L'analyse précédemment décrite ne parvenait pas à tirer d'informations intéressantes du programme \texttt{examples/analysis/condition.java} :

\begin{minted}{java}
class A {
  static int x, y;
  static boolean b1, b2;

  static void main() {
    x = Support.random(0, 4);
    if (x < 2) {
      x = x + 1;
    } else {
      x = x - 1;
    }
  }
}
\end{minted}

En faisant une disjonction de cas, on se rend compte que la variable \texttt{x} ne peut être égale, en sortie de programme, qu'à 1 ou à 2. Pour que l'analyseur soit capable de le déduire, l'analyse des structures a été affinée et chaque bloc de la structure est analysé \textit{relativement} au fait que la condition est vraie ou fausse. Plus clairement, le premier bloc est évalué comme si la condition était vraie, et le deuxième comme si la condition était fausse. Les deux environnements que l'on déduit ensuite sont fusionnés pour donner l'environnement final en sortie de structure :

\begin{verbatim}
$ ./analyzer --domain 1 examples/analysis/condition.java
FINAL STATE IS:
0 -> int: [1, 2]
1 -> int: [-infty, +infty]
2 -> bool: undetermined
3 -> bool: undetermined
----------
Class name: A
Class body:
   Declaration of variable:
     Variable:
       name: x | type: int | uniqueId: 0
   Declaration of variable:
     Variable:
       name: y | type: int | uniqueId: 1
   Declaration of variable:
     Variable:
       name: b1 | type: bool | uniqueId: 2
   Declaration of variable:
     Variable:
       name: b2 | type: bool | uniqueId: 3
   Declaration of function:
     Proc name: main
     Proc body:
        x <- random(0, 4)
        if ((x < 2)) then:
            x <- (x + 1)
        else:
            x <- (x - 1)
------------
\end{verbatim}

\subsection{Analyse de congruences}

Une analyse simple des congruences a été développée. Sur le fichier d'exemple\newline \texttt{examples/analysis/loop\_congruence.java}, l'analyse détecte que la variable \texttt{x} reste paire tout au long de l'exécution, et peut donc se débarrasser d'une partie du programme :

\begin{verbatim}
$ ./analyzer --domain 2 examples/analysis/loop_congruence.java
FINAL STATE IS:
0 -> int: = 0 [2]
1 -> int: = 0 [1]
2 -> bool: false
3 -> bool: undetermined
----------
Class name: A
Class body:
   Declaration of variable:
     Variable:
       name: x | type: int | uniqueId: 0
   Declaration of variable:
     Variable:
       name: y | type: int | uniqueId: 1
   Declaration of variable:
     Variable:
       name: b1 | type: bool | uniqueId: 2
   Declaration of variable:
     Variable:
       name: b2 | type: bool | uniqueId: 3
   Declaration of function:
     Proc name: main
     Proc body:
        x <- 0
        y <- 1
        while ((y < 1000)) do:
            x <- (x + 2)
            y <- (y + 1)
        b1 <- false
------------
\end{verbatim}

On remarquera au passage que j'ai pris la liberté d'ajouter au projet la gestion de l'égalité \texttt{==}, qui donne des cas de tests plus intéressants en ce qui concerne les congruences.

\subsection{Analyse couplée de congruences et d'intervalles}

Une analyse plus puissante que les précédentes consisterait à faire communiquer les domaines abstraits des intervalles et des congruences pour partager les informations. La première étape vers cette analyse a été de définir le produit réduit des deux domaines dans le module \texttt{domain\_congruences\_and\_intervals.ml}. Essentiellement, ce domaine abstrait contient à la fois les informations des deux domaines précédents. J'ai implémenté un partage utile d'informations entre les domaines pour le seul cas du test d'égalité. Ce partage consiste à appeler la fonction du domaine des congruences qui teste l'égalité entre deux informations, mais en lui fournissant en plus une information sur la longueur des intervalles en jeu, s'ils sont connus.\\

Pour le cas du fichier \texttt{examples/analysis/coupling.java}, cela donne des résultats satisfaisants :

\begin{verbatim}
$ ./analyzer --domain 3 examples/analysis/coupling.java
FINAL STATE IS:
0 -> int: (= 2 [4], [2, 4])
1 -> int: (1, [1, 1])
2 -> bool: true
3 -> bool: undetermined
----------
Class name: A
Class body:
   Declaration of variable:
     Variable:
       name: x | type: int | uniqueId: 0
   Declaration of variable:
     Variable:
       name: y | type: int | uniqueId: 1
   Declaration of variable:
     Variable:
       name: b1 | type: bool | uniqueId: 2
   Declaration of variable:
     Variable:
       name: b2 | type: bool | uniqueId: 3
   Declaration of function:
     Proc name: main
     Proc body:
        x <- 1
        while ((x < 100)) do:
            x <- (x + 4)
        y <- 1
        if ((x < 4)) then:
            x <- (x + 1)
        else:
            x <- 2
        b1 <- true
------------
\end{verbatim}

On peut d'ailleurs vérifier qu'aucun des deux domaines, pris séparément, ne donne autant d'informations.

\section{Conclusion}

L'analyseur semble bien fonctionner dans le cas de programmes simples. Certaines idées d'extensions peuvent encore être améliorées pour prendre en compte plus de situations courantes. On peut notamment penser à une analyse qui transforme des boucles \texttt{while} en des boucles \texttt{for} lorsque cela est possible, ce qui est très utile en vue d'optimiser la compilation après l'analyse.\\

S'il est évident qu'un analyseur doit pouvoir effectuer des analyses interprocédurales pour être réellement utile à un compilateur, c'est une fonctionnalité qui n'a pas été implémentée dans le cadre du projet. Il me semblait en effet plus intéressant de me pencher tout d'abord sur l'amélioration des analyses statiques, la prise en charge des appels de fonctions relevant surtout de difficultés d'implémentations et non de difficultés théoriques.

\end{document}
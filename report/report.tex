\title{INF565 Project report}
\documentclass[paper=a4, fontsize=11pt]{scrartcl}
\usepackage[utf8]{inputenc}
\usepackage[T1]{fontenc}
\usepackage{fourier}

\usepackage[francais]{babel}
\usepackage[protrusion=true,expansion=true]{microtype}	
\usepackage{amsmath,amsfonts,amsthm}
\usepackage{hyperref}
\usepackage{minted}
\usepackage{graphicx}

\usepackage{sectsty}
\allsectionsfont{\centering \normalfont\scshape}

\usepackage[nottoc, notlof, notlot]{tocbibind}

\usepackage{fancyhdr}
\pagestyle{fancyplain}
\fancyhead{}											% No page header
\fancyfoot[L]{}											% Empty 
\fancyfoot[C]{}											% Empty
\fancyfoot[R]{\thepage}									% Pagenumbering
\renewcommand{\headrulewidth}{0pt}			% Remove header underlines
\renewcommand{\footrulewidth}{0pt}				% Remove footer underlines
\setlength{\headheight}{13.6pt}

\numberwithin{figure}{section}			% Figurenumbering: section.fig#
\numberwithin{table}{section}				% Tablenumbering: section.tab#

\newcommand{\horrule}[1]{\rule{\linewidth}{#1}} 	% Horizontal rule

\title{	
		\usefont{OT1}{bch}{b}{n}
		\normalfont \normalsize \textsc{INF565 : Vérification} \\ [25pt]
		\horrule{0.5pt} \\[0.4cm]
		\huge Analyseur statique pour un sous-ensemble de Java \\
		\horrule{2pt} \\[0.5cm]
}
\author{
		\normalfont 								\normalsize
        Lo\"{i}c Gelle\\[-3pt]		\normalsize
        16 mars 2017
}
\date{}


%%% Begin document
\begin{document}
\maketitle
\section{Introduction}

Ce court rapport a pour objet de présenter le travail effectué lors du projet du cours INF565 -- Vérification. La première partie sera consacrée à la description de la mise en place du projet et des premières vérifications effectuées. La partie suivante détaillent les extensions qui ont été implémentées et leurs motivations.\\

Le code est disponible sur Github\footnote{Voir \url{https://github.com/loicgelle/inf565-static-analyzer}}. Des exemples sont fournis dans le dossier \texttt{code/examples/} du projet ; lorsque ce rapport fera référence à un fichier d'exemple, il le fera relativement à ce répertoire. Pour plus d'informations sur l'utilisation de l'analyseur et la structure du projet, on pourra se référer au fichier \texttt{README.md}.\\

\textit{\textbf{Important :} toutes les analyses statiques effectuées dans le projet -- sauf l'analyse de typage -- supposent que le programme consiste en une seule classe contenant une seule fonction. L'analyse interprocédurale n'a pas été envisagée.}

\section{Mise en place du projet et vérifications de base}

\subsection{Mise en place et premières analyses statiques}

Les premières étapes ont été l'implémentation d'une fonction d'affichage des programmes, à des fins de débogage, ainsi que d'un interpréteur. L'interprète suit simplement la structure de l'arbre syntaxique et modifie l'environnement -- une table de hashage qui associe à chaque variable sa valeur -- en conséquence.\\

L'interpréteur est capable de lever des exceptions en cas de problème de typage ou d'utilisation d'une variable non initialisée. Le programme \texttt{typing\_bad/binary\_op.java} suivant

\begin{minted}{java}
class A {
  static int x;
  static boolean b1, b2;

  static void main() {
    x = 1;
    b1 = true;
    b2 = x + b1;
  }
}
\end{minted}

renvoie une erreur à l'interprétation :

\begin{verbatim}
$ ./analyzer --interpret A examples/typing_bad/binary_op.java
examples/typing_bad/binary_op.java:8.9-15:
Interpretation error!
Cannot apply operator to non integer values
\end{verbatim}

De même, l'interpréteur échoue en cas d'utilisation d'une variable non initialisée, comme c'est le cas dans le fichier \texttt{init\_bad/condition.java} :

\begin{verbatim}
$ ./analyzer --interpret A examples/init_bad/condition.java
examples/init_bad/condition.java:6.4-8.5:
Interpretation error!
IF statement needs a boolean value to test
\end{verbatim}

Cependant, ces erreurs devraient plutôt être anticipées dès l'analyse statique. L'analyseur statique permet de détecter l'utilisation de variables non initialisées avant l'exécution :

\begin{verbatim}
$ ./analyzer examples/init_bad/condition.java
examples/init_bad/condition.java:6.8-10:
Error: use of not initialized variable
\end{verbatim}

L'analyseur détecte l'initialisation des variables lorsque les flots de contrôle sont un peu plus complexe, comme c'est le cas pour \texttt{init\_good/double\_condition.java}.\\

Une vérification statique des types a également été implémentée. Sur le programme mal typé présenté plus haut, l'analyseur détecte l'erreur de typage :

\begin{verbatim}
$ ./analyzer examples/typing_bad/binary_op.java
examples/typing_bad/binary_op.java:8.9-15:
Typing error
\end{verbatim}

\subsection{Analyses statiques plus poussées et simplifications}

La première analyse statique concernant la valeur des variables est une analyse de constantes. Elle est dans l'approche assez similaire à l'analyse des variables non initialisées. L'analyse imprime l'environnement courant avant chaque instruction et l'environnement final.

\end{document}